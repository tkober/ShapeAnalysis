\section{Meilenstein A}
In diesem ersten Meilenstein soll die Erzeugung einer einfach-verketteten Liste mit beliebig vielen Linsenobjekten analysiert werden. Insbesondere die folgenden Eigenschaften des Speichers nach dem Abarbeiten des entsprechenden Programmteils sind von Interesse:
\begin{itemize}
	\item Keine Objekte sind verloren gegangen und liegen unerreichbar auf dem Heap. Ihre Analyse soll zeigen, dass nach dem entsprechendem Programmteil auf dem Heap lediglich Linsenobjekte existieren, die von der Programmvariable \emph{LinsenTopf} aus erreichbar sind.
	\item Alle Linsenobjekte sind korrekt erzeugt, d.h eine Linse ist entweder gut oder schlecht. Niemals hat sie beide oder keine der Eigenschaften. Aus den von Ihrer Analyse erzeugten Strukturen muss hervorgehen, dass obige Eigenschaft gilt, also genau eines der beiden Attribute \emph{istGut} und \emph{istSchlecht} den Wahrheitswert true besitzt.
\end{itemize}


\subsection{Eingabestrukturen}
Da der zu untersuchende Programmabschnitt für die Erzeugung der Eingabestruktur von Meilenstein B zuständig ist, wird für dessen Analyse lediglich die leere Eingabestruktur benötigt. 


\subsection{TVP}
Der TVP-Code wurde von Hand aus dem C-Code erzeugt. Das Statement zum Setzen des \emph{istGut} Attributs wurde über eine \emph{uninterpreted} Kante gelöst. Dadurch muss die Analyse beide Fälle gleichwertig betrachten. Einmal den Fall, dass die \texttt{rand()} Funktion eine gerade Zahl liefert und den anderen Fall, dass eine ungerade Zahl zurückgegeben wird. Dadurch sind die \emph{gut}- und \emph{schlecht}-Eigenschaften immer genau definiert.

An den Schluss des Algorithmus wurden noch zwei weitere Anweisungen eingefügt, um die, für die Analyse irrelevanten, Variablen \emph{linse} und \emph{topf} auf Null zu setzen. Dadurch können die Ausgabestrukturen besser zusammengefasst werden, wodurch weniger Ausgabestrukturen entstehen. Das Ergebnis der Analyse wird allerdings nicht verändert.

\textsnippettiny{../cendrillon_create.tvp}{20}{45}


\subsection{Zusicherungen}
Die drei Zusicherungen garantieren die Integrität der Ausgabestrukturen. Es wird geprüft,...
\begin{itemize}
	\item ... ob die Linsen alle korrekt erzeugt wurden, d.h. nur entweder gut oder schlecht sind.
	\item ... die Liste keine Invarianten hat
	\item ... kein logisches Individuum vom Listenstart aus nicht erreichbar ist.
\end{itemize}


\subsection{Ausgabestrukturen}
Folgende Ausgabestrukturen werden von der Analyse erzeugt:

\begin{itemize}[$\rightarrow$]
	\item Es wurden ausschließlich schlechte Linsen erzeugt (mind. 2, max. beliebig) \analysispageref{Create}{5}.
	\item Es wurde nur eine gute Linse erzeugt \analysispageref{Create}{6}.
	\item Es wurde eine gute Linse, anschließend beliebig viele schlechte Linsen erzeugt \analysispageref{Create}{7}.
	\item Die erste Linse ist gut, anschließend gibt es beliebig viele gute und schlechte Linsen in beliebiger Reihenfolge \analysispageref{Create}{8}.
	\item Die erste Linse ist schlecht, anschließend gibt es beliebig viele gute und schlechte Linsen in beliebiger Reihenfolge \analysispageref{Create}{9}.
	\item Es wurde eine schlechte Linse erzeugt \analysispageref{Create}{10}.
	\item Es wurde eine schlechte Linse, anschließend beliebig viele gute Linsen erzeugt \analysispageref{Create}{11}.
	\item Es wurden ausschließlich gute Linsen erzeugt (mind. 2, max. beliebig). \analysispageref{Create}{12}.
\end{itemize}
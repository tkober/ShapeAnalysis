\section{Allgemeiner Aufbau der Analyse}


\subsection{Kernprädikate}
Zur Durchführung der Analysen mussten drei zusätzliche Prädikate definiert werden. Diese modellieren Kerneigenschaften der logischen Individuen und sollen im Folgenden genauer beschrieben werden.

\subsubsection{\emph{istGut} und \emph{istSchlecht}}
Die Prädikate \textbf{istGut} und \textbf{istSchlecht} modellieren die \emph{gut}-, bzw. \emph{schlecht}-Eigenschaften einer Linse. Ist eines dieser Prädikate true muss das Gegenstück false sein, ansonsten ist die Linse inkonsistent, gemäß Aufgabenstellung. Des Weiteren sollen für jede Linse beide Eigenschaften wohl definiert sein, d.h. nicht auf 1/2 stehen.

\textsnippet{../predicates.tvp}{16}{17}

\subsubsection{\emph{isArbitrary}}
Dieses Prädikat gibt an, dass ein logisches Individuum eine beliebige Speicherzelle modelliert. Dadurch kann ermittelt werden, ob sich eine solche Zelle in einer Ausgabestruktur befindet, wie es vorkommen kann, wenn eine Variable nicht initialisiert wird.

\textsnippet{../predicates.tvp}{19}{19}


\subsection{Actions}
Actions werden dazu genutzt die logischen Strukturen zu aktualisieren.

\subsubsection{\emph{Set\_Ist\_Gut\_False} und \emph{Set\_Ist\_Gut\_True}}
Mit diesen beiden Actions kann die \emph{gut}-Eigenschaft einer Linse gesetzt werden. Dazu wird zunächst geprüft, ob das aktuelle logische Individuum das gesuchte ist (auf das die übergebene Variable zeigt) und dieses dann auf true/false gesetzt. Bei allen anderen logischen Individuen wird der vorherige Wert beibehalten.

Bei der \emph{Set\_Ist\_Gut\_False} Action könnte der \textbf{(lhs(v) \& 0)} Teil weggelassen werden. Zur besseren Lesbarkeit wird dieser allerdings beibehalten.


\textsnippet{../actions.tvp}{19}{31}


\subsubsection{\emph{Set\_Ist\_Schlecht\_Opposite}}

\emph{Set\_Ist\_Schlecht\_Opposite} setzt den Wert der \emph{schlecht}-Eigenschaft für ein angegebenes logisches Individuum auf das Gegenteil der \emph{gut}-Eigenschaft. Dazu wird geprüft, ob das aktuelle logische Individuum das gesuchte ist, wenn ja erhält es den gegenteiligen Wert der \emph{gut}-Eigenschaft, insofern es nicht die \emph{arbitrary}-Eigenschaft hat. Andernfalls behält es seinen Wert für \emph{istSchlecht}.

\textsnippet{../actions.tvp}{33}{38}


\subsubsection{\emph{Is\_Good} und \emph{Is\_Not\_Good}}
Diese beiden Actions werden für Verzweigungen genutzt. Die Kante wird genommen, falls das angegebene logische Individuum die \emph{gut}-Eigenschaft erfüllt, bzw. nicht erfüllt.

\textsnippet{../actions.tvp}{40}{50}

\subsection{Zusicherungen}
Nach jeder Analyse werden einige Eigenschaften der Ausgabestrukturen geprüft, um deren Integrität festzustellen. Dafür werden folgende Zusicherungen (Assertions) genutzt.

\subsubsection{\emph{Asert\_No\_Good\_n\_Bad}}
Stellt sicher, dass kein logisches Individuum denselben Wahrheitswert für die \emph{gut}- und \emph{schlecht}-Eigenschaft hat (ausgenommen sind \emph{arbitrary} logische Individuen, da diese keine Linsen modellieren!).

\textsnippet{../actions.tvp}{91}{95}

\subsubsection{\emph{Asert\_No\_Reachable\_Arbitrary}}
Stellt sicher, dass kein \emph{arbitrary} logisches Individuum von einem Programmzeiger aus erreichbar ist. Falls eins erreichbar ist, wurde eine Programmvariable nicht initialisiert, was zu einem Fehler führen könnte.

\textsnippet{../actions.tvp}{67}{71}

\subsubsection{\emph{Asert\_Good\_Linsen}}
Stellt sicher, dass von einem Programmzeiger aus (in diesem Fall sollte nur der Programmzeiger \emph{guteLinsen} von Bedeutung sein) nur gute Linsen bzw. logische Individuen mit der \emph{gut}-Eigenschaft erreichbar sind.

\textsnippet{../actions.tvp}{73}{77}

\subsubsection{\emph{Asert\_Bad\_Linsen}}
Stellt sicher, dass von einem Programmzeiger aus (in diesem Fall sollte nur der Programmzeiger \emph{schlechteLinsen} von Bedeutung sein) nur schlechte Linsen bzw. logische Individuen mit der \emph{schlecht}-Eigenschaft erreichbar sind.

\textsnippet{../actions.tvp}{79}{83}

\subsubsection{\emph{Asert\_No\_Leak\_End}}
Stellt sicher, dass ein logisches Individuum nicht von dem guteLinsen und dem schlechteLinsen Zeiger aus erreichbar ist und, dass kein logisches Individuum, das nicht die \emph{arbitrary}-Eigenschaft erfüllt, von keinem dieser beiden Zeigern aus erreichbar ist.

Diese Zusicherung sichert die Integrität der guteLinsen und schlechteLinsen Ausgaben.

\textsnippet{../actions.tvp}{85}{89}

\section{Meilenstein C}
Im Quellcode ist eine alternative, elegantere Implementierung der Linsentopferzeugung (auskommentiert) vorhanden. Ersetzen Sie Ihre, bereits verifizierte Implementierung durch die Kürzere und stellen Sie sicher, dass Ihre Analyse weiterhin die Verifikationsziele der Meilensteine A und B erfüllt.


\subsection{Eingabestrukturen}
Da für diesen Meilenstein auch die partielle Korrektheit von Meilenstein B gezeigt werden soll, werden dieselben Eingabestrukturen verwendet\analysispageref{C}{3}.


\subsection{TVP}
Für den TVP-Code gilt das gleiche wie in Meilenstein A und B. Wobei die Schleife, beginnend an Programmpunkt L2, mindestens einmal durchlaufen werden muss, da die Bedingung in jedem erfüllt wird (i = 1 und n >= 1). Dies wird dadurch modelliert, dass bei der ersten Iteration die Abbruchbedingung durch ein \emph{uninterpreted} Statement übersprungen wird.
\textsnippettiny{../cendrillon_C.tvp}{18}{61}


\subsection{Zusicherungen}
Die Analyse muss die selben Zusicherungen erfüllen, wie Meilenstein B.


\subsection{Ausgabestrukturen}
Diese Analyse erzeugt die folgenden Ausgabestrukturen, welche mit denen aus Meilenstein B übereinstimmen:
\begin{itemize}[$\rightarrow$]
	\item Es sind nur gute Linsen im Topf gewesen, mindestens zwei.\analysispageref{C}{14}
	\item Es waren mindestens zwei gute und genau eine schlechte Linse im Topf.\analysispageref{C}{15}
	\item Es waren mindestens zwei schlechte und genau eine gute Linse im Topf.\analysispageref{C}{16}
	\item Es war nur eine gute Linse im Topf.\analysispageref{C}{17}
	\item Es war nur eine schlechte Linse im Topf.\analysispageref{C}{18}
	\item Es waren sowohl mindestens zwei gute als auch mindestens zwei schlechte Linse im Topf.\analysispageref{C}{19}
	\item Es sind nur schlechte Linsen im Topf gewesen, mindestens zwei.\analysispageref{C}{20}
	\item Es waren sowohl genau eine gute als auch genau eine schlechte Linse im Topf.\analysispageref{C}{21}
\end{itemize}